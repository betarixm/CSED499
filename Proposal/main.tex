\documentclass{article}
\usepackage[utf8]{inputenc}
\usepackage{kotex}
\usepackage{color}
\usepackage{fancyhdr}
\usepackage{geometry}
\geometry{
    a4paper,
    total={170mm,257mm},
    left=20mm,
    top=20mm,
}

\lhead{\large 2022년 봄 POSTECH 컴퓨터공학과 과제연구 연구계획서\\[1.3cm]}
%\lhead{\parbox[][\headheight][t]{2cm}{\textbf{Citation:}}}
%\rhead{\parbox[][\headheight][b]{2cm}{\raggedleft Page\,\thepage{} of \pageref{LastPage}}}
\rhead{\includegraphics[width=2.5cm]{logo}}
\renewcommand{\headrulewidth}{0pt}
%\title{[제목]\hfill\includegraphics[width=2cm]{logo}}
\title{[제목]}
\author{이름:\\학번:\\연구지도교수:}


\date{}

\usepackage{natbib}
\usepackage{graphicx}




\begin{document}
\maketitle\thispagestyle{fancy}

\noindent\makebox[\linewidth]{\rule{\textwidth}{0.4pt}}

\begin{center}
\textcolor{red}{* 아래 사항들에 대해 3-4페이지 내외로 명료하게 서술한 뒤 pdf로 저장하여 제출합니다}
\end{center}

\section{연구 목적}
연구의 주제를 서술하고 그 중요성과 필요성을 설명합니다.

\section{연구 배경}
연구의 배경을 서술하여 계획하는 연구가 가지는 맥락을 밝힙니다. 

\section{연구 방법}
연구 수행 방법을 간략히 서술합니다.

\section{기대 효과}
연구 결과를 통해 기대할 수 있는 학계 또는 산업계의 파급 효과를 간략히 서술합니다. 

\section{연구 추진 일정}
최종 발표에 맞추어 개략적인 연구 추진 계획을 기록합니다.


\bibliographystyle{plain}
\bibliography{references}
\end{document}
